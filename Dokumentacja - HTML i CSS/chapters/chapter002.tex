\chapter{Opis założeń projektu}
\label{cha:opisZałożeńProjektu}

%---------------------------------------------------------------------------

\section{Założenia projektu}
\label{sec:Założeniaprojektu}

Realizacja projektu \textbf{Kroniki Hogwartu} opierała się na szeregu założeń funkcjonalnych oraz wizualnych, mających na celu stworzenie witryny atrakcyjnej dla odbiorcy jak i poprawnej technologicznie. Założenia obejmowały:

\begin{itemize}
    \item \textbf{Profil odbiorcy i cel informacyjny} – strona skierowana jest do fanów uniwersum Harry'ego Pottera, którzy poszukują wiedzy podanej w atrakcyjnej i wizualnej formie.
    \item \textbf{Model witryny} – ograniczenie architektury wyłącznie do warstwy prezentacyjnej (Front-end). Projekt jest stroną statyczną, niewymagającą zaplecza serwerowego (Back-end) ani baz danych, co pozwala na łatwy hosting i szybkie działanie.
    \item \textbf{Wizualizm} – projekt zakłada, że design ma budować klimat. Strona ma pasować wizualnie do tego co się na niej znajduje, nawet jeśli oznacza to użycie bardziej złożonych efektów graficznych czy niestandardowych czcionek.
\end{itemize}

\hfill
\section{Wymagania funkcjonalne}
\label{sec:Wymaganiafunkcjonalne}
Wymagania funkcjonalne określają zakres operacji, jakie użytkownik może wykonać w obrębie witryny, oraz sposób zachowania się systemu w odpowiedzi na te działania.

\begin{itemize}
    \item \textbf{Intuicyjna nawigacja (Global Navigation)} – strona zawiera stały pasek menu (\textit{Sticky Header}), umożliwiający szybkie przemieszczanie się pomiędzy głównymi sekcjami (Strona Główna, Domy, Filmy) bez konieczności przewijania na początek strony.
    \item \textbf{Prezentacja treści multimedialnych} – witryna umożliwia przeglądanie szczegółowych informacji o serii filmowej, w tym opisów fabuły, głównych bohaterów a także charakterystyki Domów Hogwartu z wykorzystaniem dedykowanych grafik.
    \item \textbf{Integracja z zasobami zewnętrznymi} – zaimplementowano aktywne odnośniki do mediów społecznościowych (Facebook, Instagram, X), zewnętrznych źródeł wiedzy (Wikipedia) oraz serwisów wideo.
    \item \textbf{Mechanizm powrotu} – na każdej podstronie zaimplementowano funkcjonalność, pozwalającą użytkownikowi na natychmiastowe przewinięcie widoku do początku okna przeglądarki.
\end{itemize}

\newpage
\section{Wymagania niefunkcjonalne}
\label{sec:Wymaganianiefunkcjonalne}
Wymagania niefunkcjonalne definiują atrybuty jakościowe systemu, ograniczenia technologiczne oraz standardy, jakimi kierowano się podczas procesu tworzenia.

\begin{itemize}
    \item \textbf{Responsywność (RWD)} – witryna musi poprawnie skalować się na wszystkich typach urządzeń, w szczególności dla szerokości 900px, poniżej której układ, dostosowany jest do ekranów dotykowych.
    \item \textbf{Kompatybilność przeglądarkowa} – kod HTML i CSS został zoptymalizowany pod kątem poprawnego wyświetlania w najpopularniejszych przeglądarkach internetowych (Google Chrome, Mozilla Firefox, Microsoft Edge, Safari).
    \item \textbf{Estetyka i spójność wizualna} – interfejs musi utrzymywać jednolity styl graficzny. Elementy strony muszą charakteryzować się odpowiednim doborem efektów, zachowując przy tym wysoki kontrast tekstu dla zapewnienia czytelności.
    \item \textbf{Czystość technologiczna} – całość warstwy wizualnej i animacji interakcji oparta jest wyłącznie na kaskadowych arkuszach stylów (CSS3).
\end{itemize}