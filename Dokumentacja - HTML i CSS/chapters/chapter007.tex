\chapter{Podsumowanie}
\label{sec:summary}

Realizacja projektu \textbf{Kroniki Hogwartu} zakończyła się sukcesem. Wszystkie założone cele funkcjonalne oraz wizualne zostały osiągnięte. Powstała witryna jest w pełni responsywna, estetyczna i zgodna z nowoczesnymi standardami webowymi, co potwierdziły przeprowadzone testy walidacyjne.

Praca nad projektem bez użycia gotowych frameworków (takich jak Bootstrap) okazała się cennym doświadczeniem. Pozwoliła na głębokie zrozumienie mechanizmów działania kaskadowych arkuszy stylów, w szczególności modułów \texttt{Grid} oraz \texttt{Flexbox}. Ręczne implementowanie responsywności (Media Queries) dało znacznie większą kontrolę nad wyglądem strony na urządzeniach mobilnych niż korzystanie z gotowych szablonów.

Stworzona strona internetowa stanowi solidną bazę do dalszego rozwoju. W przyszłości projekt można rozbudować o warstwę skryptową (JavaScript), która dodałaby interaktywne elementy, takie jak dynamiczne galerie zdjęć czy quizy wiedzy o uniwersum. 

Projekt udowadnia, że przy użyciu wyłącznie technologii HTML5 i CSS3 można stworzyć nowoczesną, atrakcyjną wizualnie i w pełni funkcjonalną witrynę internetową.