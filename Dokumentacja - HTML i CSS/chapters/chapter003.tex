\chapter{Opis struktury projektu}
\label{cha:opisStrukturyProjektu}

\section{Struktura projektu}
\label{sec:Strukturaprojektu}

Architektura projektu została zaprojektowana z wyraźnym podziałem na warstwę strukturalną (HTML) oraz prezentacyjną (CSS). Taka organizacja plików zapewnia łatwość w zarządzaniu kodem oraz jego dalszym rozwoju.\newline

\section{Organizacja plików}
\label{sec:Organizacjaplików}

Projekt składa się z powiązanych ze sobą plików, które tworzą spójną całość witryny:

\begin{itemize}
    \item \texttt{strona.html} – plik startowy, pełniący rolę strony głównej. Zawiera podstawowe informacje o serii oraz opis głównych bohaterów.
    \item \texttt{domy.html} – podstrona prezentująca cztery domy Hogwartu. Zawiera informacje na temat każdego z poszczególnych domów.
    \item \texttt{filmy.html} – podstrona dedykowana filmom. Zawiera listę wszystkich części sagi filmowej wraz z opisami fabuły oraz z odnośnikami do poszczególnych filmów wideo. 
    \item \texttt{style.css} – arkusz stylów, odpowiedzialny za warstwę wizualną wszystkich powyższych podstron. Dzięki podpięciu jednego pliku CSS do wszystkich dokumentów HTML, zachowano spójność graficzną w całym projekcie.
    \item \textbf{Zasoby graficzne} – zbiór plików graficznych, obejmujący tło, herby domów, zdjęcia postaci, plakaty filmów oraz ikony interfejsu i mediów społecznościowych.\newline
\end{itemize}

\section{Architektura kodu CSS}
\label{sec:ArchitekturakoduCSS}

Plik \texttt{style.css} jest przemyślaną strukturą odpowiedzialną za wygląd strony. Jego budowę można podzielić na kluczowe elementy:

\begin{enumerate}
    \item \textbf{Zmienne globalne (:root)} – na początku pliku zdefiniowano paletę kolorystyczną (m.in. \texttt{--gold}, \texttt{--bg-glass}), co pozwala na zmianę motywu przewodniego strony w jednym miejscu.
    \item \textbf{System Grid i Flexbox} – główny układ strony oparty jest na kontenerze \texttt{.grid}, który dzieli widok na obszar menu i treści. Wewnątrz sekcji wykorzystano \texttt{Flexbox} do pozycjonowania elementów.
    \item \textbf{Responsywność} – końcowa sekcja pliku zawiera reguły dla ekranów o szerokości poniżej 900px. Nadpisują one domyślne style, zmieniając układ poziomy na pionowy oraz dostosowując wielkość czcionek i marginesów do urządzeń dotykowych.
\end{enumerate}
