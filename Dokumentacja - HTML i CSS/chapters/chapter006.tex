\chapter{Weryfikacja zgodności ze standardami}
\label{cha:Weryfikacjazgodnościzestandardami}

Kluczowym etapem finalizacji projektu było przeprowadzenie walidacji kodu. Celem tego procesu było potwierdzenie, że napisana struktura HTML oraz CSS są zgodne z obowiązującymi standardami W3C (World Wide Web Consortium). Poprawność kodu ma kluczowe znaczenie dla:

\begin{itemize}
    \item \textbf{Kompatybilności} – gwarantuje poprawne wyświetlanie strony na różnych przeglądarkach i systemach operacyjnych.
    \item \textbf{Dostępności (Accessibility)} – ułatwia interpretację strony przez czytniki ekranowe dla osób niepełnosprawnych.
    \item \textbf{SEO (Search Engine Optimization)} – semantyczny, bezbłędny kod jest lepiej indeksowany przez roboty wyszukiwarek (np. Google).\newline
\end{itemize}

\section{Walidacja HTML}
\label{sec:WalidacjaHTML5}

Do weryfikacji warstwy strukturalnej wykorzystano narzędzie \textbf{W3C Markup Validation Service}. Przeprowadzony test dla pliku \texttt{strona.html} (oraz pozostałych podstron) zakończył się wynikiem pozytywnym. Walidator nie wykrył żadnych błędów ani ostrzeżeń.

\begin{figure}[H]
    \centering
    \includegraphics[width=1.0\textwidth]{figures/test1.jpg}
    \caption{Wynik walidacji pliku \texttt{strona.html}.}
    \label{fig:htmlval}
\end{figure}

\newpage
\section{Walidacja CSS}
\label{sec:WalidacjaCSS3}

Arkusze stylów zostały poddane analizie przy użyciu narzędzia \textbf{W3C CSS Validation Service}. Sprawdzono plik \texttt{style.css} pod kątem poprawności składni oraz zgodności z profilem CSS level 3 + SVG. Test zakończył się sukcesem, potwierdzając brak błędów.

\begin{figure}[H]
    \centering
    \includegraphics[width=1.0\textwidth]{figures/test2.jpg}
    \caption{Wynik walidacji pliku \texttt{style.css}.}
    \label{fig:cssval}
\end{figure}