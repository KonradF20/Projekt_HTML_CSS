\chapter{Harmonogram realizacji projektu}
\label{cha:harmonogram}

Praca nad projektem \textbf{Kroniki Hogwartu} została podzielona na etapy, począwszy od fazy koncepcyjnej, przez implementację kodu, aż po optymalizację i testy końcowe. Poniższa tabela przedstawia szczegółowy przebieg tego procesu.\newline

\begin{table}[h!]
\centering
\renewcommand{\arraystretch}{1.5}
\begin{tabular}{|c|p{10cm}|}
\hline
\textbf{Etap} & \textbf{Opis realizowanych zadań} \\ \hline
\textbf{1. Analiza i Koncept} & Określenie tematyki, zgromadzenie niezbędnych materiałów graficznych oraz treści dotyczących fabuły i bohaterów. \\ \hline
\textbf{2. Struktura HTML} & Utworzenie szkieletu plików \texttt{.html}. Zbudowanie podziału na sekcje (kontenery \texttt{div}, nagłówki, nawigacja). \\ \hline
\textbf{3. Stylizacja (CSS)} & Implementacja zmiennych globalnych (\texttt{:root}) oraz głównego układu \textit{Grid}. Dodanie tła i innych obrazów oraz stylizacji elementów. \\ \hline
\textbf{4. Responsywność (RWD)} & Konfiguracja dla urządzeń mobilnych. Dostosowanie paska nawigacji do ekranów dotykowych oraz zmiana układu z poziomego na pionowy. \\ \hline
\textbf{5. Optymalizacja i Detale} & Dopracowanie estetyki: dodanie stopki, poprawa marginesów i odstępów, wyśrodkowanie elementów na urządzeniach mobilnych. \\ \hline
\end{tabular}
\caption{Harmonogram prac nad projektem}
\end{table}

\section{Repozytorium i system kontroli wersji}
\label{sec:Repozytoriumisystemkontroliwersji}

Do zarządzania projektem wykorzystano system kontroli wersji Git. Projekt znajduje się w repozytorium utworzonym na platformie GitHub pod adresem:

\begin{center}
\url{https://github.com/KonradF20/Projekt_HTML_CSS.git}
\end{center}

Repozytorium zawiera strukturę projektu \textbf{Kroniki Hogwartu} oraz dokumentację.